\section{Introduction}

For the last years, language processing and understanding systems have been
dominated by deep learning models which can exhibit impressive performances.
Yet, their success comes with the cost of requiring huge amounts of training
data. This may be a reasonable price to pay when the training is done "offline"
(for example, machine translation systems trained once and for all), but it may
be prohibitive to use such models for learning from human interaction, where
data is particularly scarce. In contrast, earlier work on human-machine
communication have relied on hand-coded domain-specific models.  For example,
\cite{Wang:etal:2016} describe SHRDLURN, a game where a system has to manipulate
piles of colored blocks following natural language instructions given by a
human user. While the system knows nothing about the language that the user
will use, it does encode the nature of the operations that it can execute on
this world and how they compose. While this assumption is reasonable enough for
this toy domain, in the long run it may be desirable to adopt end-to-end
systems that can learn the possible actions with no need of manual encoding. 

In this work we aim at bridging these two seemingly contradictory requirements
--fast learning on one end, and flexibility on the other-- proposing a two step
training regime. First, we train an end-to-end architecture to play SHRDLURN
using utterances drawn from a (large) artificial grammar. This step allows the
system to learn the space of possible operations in this game.  Next, we use
the pre-trained system on an online learning phase, where it has to quickly
adapt to the actual language of the current speaker, which can be very
different from the one it was trained on. We explore two different adaptation
strategies. One, following Herbelot et al. (2017), we apply plain gradient
descent on the new training instances, albeit with a relatively higher learning
rate. Second, we propose a new system architecture that can choose to re-use
previously learned word embeddings as an educated guess for what a new word
could mean.  
